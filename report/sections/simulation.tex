%!TEX root = ../main.tex
\section{Simulation}\label{sec:Simulations}
Simulation tools have been a great help, not only in understanding the electromechanical system, and design a controller from it, but also in estimating requirements for the electric components.
Both Simscape and Plecs Standalone have been used, as they excel at different purposes. 
Simscape is used to test the controller on a system with as much realism as possible to make sure the go-kart works intuitively, whereas Plecs is used to simulate the inverter to determine whether there will be heating issues during operation.

Simscape will be used to simulate how the controller handles certain events:

\begin{itemize}
	\item Actuation of torque pedal.
	\item Speed limiting due to voltage limit
	\item Release of torque pedal.
	\item Actuation of brake pedal - with and without wheel lock
	\item Wheelspin
\end{itemize}

In all cases, the controller should ensure that the motor produces the requested torque.
The kart does not have regenerative braking, so negative torque would be a problem.
The Plecs simulations do not need these details, but the mechanical system still needs a representation of mass of the car, the gear, the wheel and wind resistance. 

\subsection{Mechanical System}\label{sub:Simulations_mec}
The mechanical system consists of a mass of the go-kart with driver, turbulent air resistance, wheels and a gear.

\begin{figure}[H]
	\centering
	\includegraphics[width=15cm]{graphics/simulations_mechanical_full.png}
	\caption{Block diagram of the mechanical system}
	\label{fig:mechanical_full}
\end{figure}

Figure~\ref{fig:mechanical_full} shows the block diagram of the mechanical part, that the motor will drive.
\todo{Thomas: So, a model of the kart? Martin: Yeah a very simple one.}
Starting from the left, the Permanent Magnet Synchronous Motor is the motor and the Ideal Rotational Motion Sensor is used for the Clarke-Park transformations. 
The first inertia block contains both the inertia of the motor, which is 0.0052, and the motor-side gear. 
Inertias on the same rod can be added together, so instead of having an inertia for the motor, and one for the motor-side gear, one inertia is enough
The inertia of the gear depends on its size. 
Assuming, that the gear is a disc, the mass is calculated by:

\begin{equation}
m_{G1} = \rho \pi r^2 \cdot h
\label{eq:mass of disc}
\end{equation}

where $\rho$ is the mass density of iron of $7870 \dfrac{kg}{m^3}$, r is the radius, and h is the thickness of 7 mm.
Radius is defined by the number of teeth of the gear, G, and the pitch, which is the distance between two adjacent teeth, which is 12.5 mm.
Hence the radius can be calculated by equation~\ref{eq:radius_from_G}
\begin{equation}
	r=\frac{G \cdot d_p}{2 \pi}
	\label{eq:radius_from_G}
\end{equation}

where $\mathrm{d_p}$ is the pitch, and G is the number of teeth of the gear. \\

Inertia of a disc depends on mass and radius of the disc according to equation~\ref{eq:inertia_of_disc}.

\begin{equation}
J_G = \frac{mr^2}{2} = \frac{\rho \pi r^2 \cdot h \cdot r^2}{2} = G^4 \frac{\rho \pi \frac{pitch}{2 \pi} \cdot h}{2} \approx 1.36 \cdot 10^{-9} G^4
\label{eq:inertia_of_disc}
\end{equation}

This equation is used to dertermine the inertia of the two cogs, the motor-side cog, G1, has 12 teeth, and the wheel-side cog, G2, has 50 teeth. 
This ratio, G2/G1 is put into the block "Gear Box". \\

The mass is set to 150 - 250 kg, depending on the driver, and this includes the mass of the car.
This is done to test different situations -- a high mass would have more stable speeds, but also cause more power loss in the inverter, whereas a lower mass would be more prone to wheel spin.\\

Win resistance has a significant effect on the speed of the kart, especially at higher speeds.
Wind resistance is of a gokart is primarily turbulent, and can be calculated by equation~\ref{eq:wind_resistance}.

\begin{equation}
\label{eq:wind_resistance}
F=-\frac{1}{2} \rho A c v^2
\end{equation}

where $\rho$ is the density of air, $A$ is the frontal area, $c$ is the drag coefficient and $v$ is the speed. 
The frontal area has been approximated to two boxes with the combined area of 0.6\si{\metre\squared}.\todo[inline]{Martin: add a picture of a go kart with two boxes across it, the lower one being 1 m wide, and 0.4 m high, and the second one being 0.5 m wide an 0.4 m high.} 
The density of air is approximately 1.225 \si{\kilogram\per\metre\cubed}, $c$ is approximately 0.8, according to a paper about air resistance found online. \todo[inline]{Thomas: Ofcourse you have a reference to this paper? If not, find one and add it to the bib.tex file. Martin: Yes, but it's a long address, I don't know where I'd put it because we literally only use one number: http://www.torvergata-karting.it/filemanager/download/191/The\%20evaluation\%20of\%20aerodynamic\%20drag\%20of\%20go-karts\%20by\%20means\%20of\%20coast\%20down\%20test\%20and\%20CFD\%20analysis.pdf}
The constants are multiplied into one constant called $c\_drag$, as shown in equation~\ref{eq:cdrag}

\begin{equation}
F=c_{drag} v^2 = -0.296 v^2
\label{eq:cdrag}
\end{equation}

These constants are put into the gain block, "Drag coefficient", and multiplied by the square of the speed. 
The result is put into an ideal force source as seen on figure~\ref{fig:mechanical_full}. 

This system can be simplified, so that all inertia and mass is combined in one block and the gear and wheel can be removed. 
This is done by a set of rules that apply for this mechanical circuit: This gear box reduces the speed, and increases torque, much like a transformer reduces voltage and increases current.. 
When converting from linear to rotational mechanics, torque is force times radius of the wheel, and speed is angular velocity times radius.
To turn the mass, m, into an inertia, assume a cylindrical shell with radius r, and mass m. Inertia is then calculated by:

\begin{equation}
J=mr^2
\end{equation}

This inertia can then be added to the inertia $J\_G2$. Same rules apply for a gear as for a transformer when reflecting a load from one side to the other, as shown in equation~\ref{eq:inertia_reflect}

\begin{equation}
\label{eq:inertia_reflect}
J_{ref} = \frac{G1^2}{G2^2} J
\end{equation}

This inertia is then added to the inertia of the motor and $\mathrm{J_{G1}}$:

\begin{equation}
J = (mr^2+J_{G2}) \cdot \big(\tfrac{G1}{G2}\big)^2 + J_{G1}+J_M
\end{equation}

For a mass of 250 kg, this comes to 0.282\si{\kilogram\metre\squared}.

In equation~\ref{eq:cdrag}, speed can be replaced with angular velocity and a gain, and force can be replaced by torque and a gain. So the equation becomes this:

\begin{equation}
\frac{T G2}{r G1} = c_{drag} \big(\omega r \tfrac{G1}{G2}\big)^2
\end{equation}

Isolating $T$, we get:
\todo{Thomas: Poor omega :'(. Martin: I like your humor}
\begin{equation}
T= c_{drag} \Big(\frac{G1 r}{G2}\Big)^3 \omega^2 \approx -11.1\cdot 10^{-6} \omega^2
\end{equation}

The mechanical diagram is reduced to figure ~\ref{fig:reduced_mechanical_system}

\begin{figure}[H]
	\begin{center}
	\includegraphics[width=12cm]{graphics/simulations_mechanical_simplified.png}
	\caption{Block diagram of the reduced mechanical system}
	\label{fig:reduced_mechanical_system}
	\end{center}
\end{figure}

This can be ported to Plecs, where all the used mechanical parts exist. 
Only difference is, that the inertia block is placed inline with the wire, and not as an appendage.
\todo{Thomas: These are technical details that are completely irrelevant in relation to the simulations. Martin: Yeah but it does look different.}
Finally, for the Simscape model, the tire is modelled by a friction block in series with the rotor, and the brake is modelled in a subsystem, as seen on figure~\ref{fig:simulations_mechanical_simscape}.

\begin{figure}[H]
	\begin{center}
		\includegraphics[width = 12cm]{graphics/simulations_mechanical_simscape.png}
		\caption{Block diagram for the mechanical system used in Simscape}
		\label{fig:simulations_mechanical_simscape}
	\end{center}
\end{figure}

The tire block consists of static and dynamic.
Tyre slip is a comprehensive area of study, so the rotational friction block isn't necessarily accurate. 
However, it does enable sudden changes in motor speed and load, similar to spinning and locking the wheels
The brake subsystem is a PI controlled ideal torque source, which will attempt to bring the go kart to a halt when the brake is pressed.
Without a PI controller, there is a strong possibility that the brake would suddenly make the motor go backwards.
Lastly the inertia is split into the inertia of the motor and gears on the left of the tyre, and the inertia due to the mass of the car on the right.
Since the brake block is still connected directly to the rod of the motor, a gain block is used to scale down the torque through the gear. 

\subsection{Motor Model}\label{sub:motor_model_simscape}
The Permanent Magnet Synchronous Motor is found in Simscape $\rightarrow$ SimPowerSystem $\rightarrow$ Simscape Components $\rightarrow$ Machines $\rightarrow$ Permanent Magnet Rotor. 
\todo[inline]{Thomas: Irrelevant}
This is a quite simple model, with only four parameters: Number of pole pairs, flux linkage of the magnet, inductance and armature resistance. 
A similar block can be found in Plecs under Electrical $\rightarrow$ Machines. 
\todo[inline]{Thomas: Again, Irrelevant}
This model has the same parameters as in Simscape, but also inertia and friction.

\begin{table}[h]
	\centering
	\begin{tabular}{| S | S | S |}
		\hline
		{Simscape parameters} & {Plecs parameters} & {Value} \\
		\hline
		{Permanent magnet flux linkage} & {Flux induced by magnet Phi} & {1.83225e-2} \\
		\hline
		{Stator Inductance, Ld, Lq, L0} & {Stator inductance Ld Lq} & {4e-5}\\
		\hline
		{Stator resistance per Phase, Rs} & {Stator resistance R} & {6.5e-3}\\
		\hline
		{Number of pole pairs} & {Number of pole pairs p} & {4}\\
		\hline
	\end{tabular}
	\caption{Parameters used in simulations.}
	\label{tab:motor_parameters_in_simulations}
\end{table}

One thing to note in both cases is that the flux linkage is divided by the number of pole pairs.
The reason for this is likely a matter of definitions, and the relation has been deduced using simulations. 
Armature resistance and inductances are per-phase, and the values used are found in section~\ref{sub:1117_param}.

\subsection{Electrical Network and control}\label{sub:sim_electrical}
This is where the Simulink and Plecs block diagram differ a lot. 
The purpose of using Simulink is that it is quick and easy to change multiple parameters in order to develop and test a controller. 
The advantage in Plecs is its ability to simulate switch mode power electronics, where there is a vast ratio between the minimum timestep defined by the switching frequencies, and the duration of the simulation. \todo[inline]{I am not sure i understand why this ratio is important? Martin: Different simulators have different methods. Plecs simulates right before and right after a switching event, that means 4 time steps per period per switch. Highside and lowside transistors switch at the same time, so a three phase inverter at 20 kHz means the whole circuit needs to be solved (at least) 240.000 times per second. Simulink does not solve switches as easy, so the number is significantly higher.}
The sparse electrical network along with the discrete controller and modulation blocks have been shown in figure~\ref{fig:simulations_electrical}.

\begin{figure}[h]
	\begin{center}
	\includegraphics[width=16cm]{graphics/simulations_electrical.png}
	\caption{The Simulink electrical network and modulation.}
	\label{fig:simulations_electrical}
	\end{center}
\end{figure}

The motor block has an external connection to neutral, which the real motor doesn't have. 
This neutral seems to need a dc path to ground, and so does the controlled voltage sources. 
Since the external ground cannot be connected to the internal star point of the motor, the connection is made with a very large resistor of 1\si{\giga\ohm}. 
The lighter blue wire going into the "\texttildelow" port of the PMSM block is a three phase electrical cable, which is used throughout the SimPowerSystem sublibrary, and the Splitter collects three wires into a cable. \\

Current is sensed on wires A and B, and used to calculate $I_C$ in the Zybo block.

The angular position is measured with an ideal position sensor, and then sent to the encoder block.
Here, the finite precision of the encoder is simulated by equation~\ref{eq:Encoder_block_function}

\begin{equation}
\label{eq:Encoder_block_function}
output = \left\lfloor \Bigg( \frac{\phi \cdot 256}{2 \pi} + 0.5 \Bigg) \% 256\right\rfloor
\end{equation}

where \% is the mod function. 
The purpose of that is to wrap \todo{Wrap is the common way of saying, if a number goes outside a range, it restarts from the other end of the range, like an integer overflow is (ofteh) handled in coding.} the output to a value between 0 and 255, which can be used for look-up tables. 
The round down function rounds a number, effectively quantizing the output. 
The reason for using round down rather than just round is to ensure, that the number ranges from 0 to 255. 
To combat the inaccuracy of the round down function, 0.5 is added. 
The parking test in section~\ref{sub:parking_test}, does not have accuracy of less than one, so in reality, there is likely an inaccuracy of $\mathrm{\pm 1}$
It has been attempted to use the quantizer block, but that causes stiffness to the point where the simulations almost stall. 
The output is sent to the Zybo block, which will be explained in section~\ref{sub:sim_zybo}.\\

The Zybo generates duty cycles for each phase ranging from -1 to 1. 
This value is then multiplied with half the DC voltage, and saturated, of the ideal voltage sources do not provide more voltage than the battery can.
\todo[inline]{Thomas: I don't understand this last paragraph?}
\subsubsection{Zybo block}\label{sub:sim_zybo}
The Zybo block consists of three other subsystems, corresponding with some of the blocks on the actual Zybo; Clarke-Park, Discrete controller and PWM generation.
This can be seen on figure~\ref{fig:simulations_zybo}.

\begin{figure}[H]
	\begin{center}
		\includegraphics[width = 12cm]{graphics/simulations_zybo}
		\caption{Block diagram describing the digital part of the system}
		\label{fig:simulations_zybo}
	\end{center}
\end{figure}

The Clarke-Park and PWM generation blocks are running in variable time steps, and the Discrete controller runs with a fixed step of 0.1 \si{\milli\second}.
The signal builder block is used to shape the torque requested by the pedal.
This torque is converted into the current used in the controller by multiplying with $\tfrac{3}{2 KT_T}$. 
The Clarke-Park block converts $\mathrm{I_{AB}}$ to $\mathrm{I_{dq}}$. 

\begin{equation}
\left[ \begin{array}{c}
	I_d \\ I_q
\end{array} \right]
=
\frac{2}{3}
\begin{bmatrix}
\cos (\Phi) & \cos(\Phi - \gamma) & \cos( \Phi + \gamma) \\
-\sin (\Phi) & -\sin(\Phi - \gamma) & -\sin( \Phi + \gamma)
\end{bmatrix}
\cdot
\begin{bmatrix}
I_A \\
I_B \\
I_C
\end{bmatrix}
\end{equation}

where $\gamma = \tfrac{2 \pi}{3}$. 
Since the encoder only gives 64 different angles, it is possible to do this with six lookup tables, without any loss of precision.
\todo[inline]{Thomas: Not understood}
The same lookup tables are used in the PWM generator block

\subsection{Plecs model}\label{sub:sim_plecs_electrical}
As previously mentioned, the Plecs model differs vastly from the Simulink model in the electrical network, as it more closely resembles the real analog circuit. 
It is shown on figure~\ref{fig:plecs_electrical}.
\todo[inline]{Thomas: This needs something more.. }
\begin{figure}[H]
	\begin{center}
		\includegraphics[width = \textwidth]{graphics/Plecs_electrical.pdf}
		\caption{Block diagram for the svm plecs simlations.}
		\label{fig:plecs_electrical}
	\end{center}
\end{figure}


