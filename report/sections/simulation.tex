\section{Simulation}\label{sec:Simulations}

Simulation tools have been a great help, not only understanding the electromechanical system, and design a controller from it, but estimating requirements for the electric components. Primarily, Simscape has been used, but Plecs has also been used specifically to simulate the inverter. This section will explain how the model has been build, starting with the mechanical system, then the motor model, the transformation and control part, and finally the three-phase switching pole. 

\subsection{Mechanical System}\label{sub:Simulations_mec}
The mechanical system consists of a mass of the gokart with driver, turbulent air resistance, wheels and a gear. While there are a lot of interesting blocks in the Simscape $\rightarrow$ SimDriveline library, we will stick with the mechanical library, because this is detailed enough, and more easy to port to Plecs. 

\begin{figure}[H]
	\includegraphics[width=15cm]{graphics/simulations_mechanical_full.png}
	\caption{Block diagram of the mechanical system}
	\label{fig:mechanical_full}
\end{figure}

Figure~\ref{fig:mechanical_full} shows the block diagram of the mechanical part, that the motor will drive. Starting from the left, the Permanent Magnet Synchronous Motor is the motor and the Ideal Rotational Motion Sensor is used for the Clarke-Park transformations. The first inertia block contains both the inertia of the motor, which is 0.0052, and the motor-side gear. It is possible to add inertias that are connected to the same mechanical rod. The inertia of the gear depends on its size. Assuming, that the gear is a disc, the mass is calculated by equation~\ref{eq:mass of disc}:

\begin{equation}
m_{G1} = \rho \pi r^2 \cdot h
\label{eq:mass of disc}
\end{equation}

where rho is the mass density of iron of $7870 \dfrac{kg}{m^3}$, r is the radius, and h is the thickness of 7 mm. Radius is defined by the number of teeth, G1, of the gear and the pitch, which is the distance between two adjacent teeth, which is 12.5 mm. Hence the radius can be calculated by equation. Inertia of a disc depends on mass and radius of the disc.

\begin{equation}
	r=\frac{G \cdot pitch}{2 \pi}
	\label{eq:radius_from_G}
\end{equation}

\begin{equation}
J_G = \frac{mr^2}{2} = \frac{\rho \pi r^2 \cdot h \cdot r^2}{2} = G^4 \frac{\rho \pi \frac{pitch}{2 \pi} \cdot h}{2} \approx 1.36 \cdot 10^{-9} G^4
\label{eq:inertia_of_disc}
\end{equation}

This equation is used to dertermine the inertia of the two cogs, the motor-side cog, G1, has 12 teeth, and the wheel-side cog, G2, has 50 teeth. This ratio, G2/G1 is put into the block "Wheel and Axle".

The mass is set to 200 -- 250 kg, depending on the driver, and this includes the mass of the car. This is a high estimate. A lower value could make the controller less stable, but a higher value will heat up the inverter more. 

Turbulent friction caused by the wind resistance is calculated by equation~\ref{eq:wind_resistance}.

\begin{equation}
F=-\frac{1}{2} \rho A c v^2
\label{eq:win_resistance}
\end{equation}

where $\rho$ is the density of air, A is the frontal area, c is the drag coefficient and v is the speed. The frontal area has been approximated to two boxes with the combined area of $0.6 m^2$\todo{add a picture of a go kart with two boxes across it, the lower one being 1 m wide, and 0.4 m high, and the second one being 0.5 m wide an 0.4 m high. mba}. The density of air is approximately $1.225 \frac{kg}{m^3}$, and the constant c is approximately 0.8, according to a paper about air resistance found online. The constants are multiplied into one constant called c\_drag, as shown in equation~\ref{eq:cdrag}
%"http://www.torvergata-karting.it/filemanager/download/191/The%20evaluation%20of%20aerodynamic%20drag%20of%20go-karts%20by%20means%20of%20coast%20down%20test%20and%20CFD%20analysis.pdf"
These constants are put into the gain block, "Drag coefficient", and multiplied by the square of the speed. The result is put into an ideal force source. 

\begin{equation}
F=-c_{drag} v^2 = -0.296 v^2
\label{eq:cdrag}
\end{equation}

This system can be simplified, so that all inertia an mass become one block and the gear and wheel can be removed. This is done by a set of rules that apply for this mechanical circuit: The gear box reduces the speed, and increases torque. Torque is force times radius, and speed is angular velocity times radius.
To turn the mass, m, into a inertia, assume a cylindrical shell with radius r, and mass m. Inertia is then calculated by:

\begin{equation}
J=mr^2
\end{equation}

This inertia can then be added to the inertia J\_G2. Same rules apply for a gear as for a transformer when reflecting a load from one side to the other, as shown in equation~\ref{eq:inertia_reflect}

\begin{equation}
J_{ref} = \frac{G1^2}{G2^2} J
\label{eq:inertia_reflect}
\end{equation}

This inertia is then added to the inertia of the motor and J\_G1:

\begin{equation}
J = (mr^2+J_{G2}) \cdot \big(\tfrac{G1}{G2}\big)^2 + J_{G1}+J_M
\end{equation}

For a mass of 250 kg, this comes to $0.282 kg m^2$.

In equation~\ref{eq:win_resistance}, speed can be replaced with angular velocity and a gain, and force can be replaced by torque and a gain. So the equation becomes this:

\begin{equation}
\frac{T G2}{r G1} = c_{drag} \big(\omega r \tfrac{G1}{G2}\big)^2
\end{equation}

isolating T, and putting $\omega$ by it self, we get:

\begin{equation}
T= c_{drag} \Big(\frac{G1 r}{G2}\Big)^3 \omega^2 \approx -11.1\cdot 10^{-6} \omega^2
\end{equation}

The mechanical diagram is reduced to figure ~\ref{fig:reduced_mechanical_system}

\begin{figure}[H]
	\begin{center}
	\includegraphics[width=12cm]{graphics/simulations_mechanical_simplified.png}
	\caption{Block diagram of the reduced mechanical system}
	\label{fig:reduce_mechanical_system}
	\end{center}
\end{figure}

His can be ported to Plecs, where all the used mechanical parts exist. Only difference is, that the inertia block is placed inline with the wire, and not as an appendage.

\subsection{Motor Model}\label{sub:motor_model_simscape}
The Permanent Magnet Synchronous Motor is found in Simscape $\rightarrow$ SimPowerSystem $\rightarrow$ Simscape Components $\rightarrow$ Machines $\rightarrow$ Permanent Magnet Rotor. This is a quite simple model, with only four parameters: Number of pole pairs, flux linkage of the magnet, inductance and armature resistance. The parameters can be seen on figure~\ref{fig:pmsm_parameters_simscape}. A similar block can be found in Plecs under Electrical $\rightarrow$ Machines. This model has the same parameters as in Simscape, but also inertia and friction.

\begin{figure}[H]
	\centering
	\begin{subfigure}[t]{.49\linewidth}
		\centering
		\includegraphics[width=\textwidth]{graphics/PMSM_simulink.png}
		\label{fig:pmsm_parameters_simscape}
		\caption{Simscape PMSM parameters.}
	\end{subfigure}
	\begin{subfigure}[t]{.39\linewidth}
		\centering
		\includegraphics[width=\textwidth]{graphics/PMSM_Plecs.png}
		\label{fig:pmsm_parameters_Plecs}
		\caption{Plecs PMSM parameters.}
	\end{subfigure}
	\label{fig:pmsm_parameters_plecs_simscape}
	\caption{Comparison of Simscape and Plecs models of the Permanent Magnet Synchronous Motor.}
\end{figure} 

One thing to note in both cases is, that the flux linkage is divided by the number of pole pairs. The reason for this is likely a matter of definitions, and the relation has been deduced using simulations. Armature resistance and inductances are per-phase, and the values used are found in section~\ref{sub:1117_param}.

\subsection{Electrical Network and control}\label{sub:sim_electrical}
This is where, the Simulink and Plecs block diagram differ a lot. The purpose of using Simulink is that it's quick and easy to change multiple parameters in order to develop and test a controller. The advantage in Plecs its ability to simulate switchmode power electronics, where there's a vast ratio between the minimum timestep defined by the switching frequencies, and the duration of the simulation. The sparse electrical network along with the discrete controller and modulation blocks have been shown in figure~\ref{fig:simulations_electrical}.

\begin{figure}[h]
	\begin{center}
	\includegraphics[width=16cm]{graphics/simulations_electrical.png}
	\caption{Block diagram of the Simulink electrical network and modulation.}
	\label{fig:simulations_electrical}
	\end{center}
\end{figure}

The motor block has an external connection to neutral, which the real motor doesn't have. This neutral seems to need a dc path to ground, and so does the controlled voltage sources. Since the external ground cannot be connected to the internal star point of the motor, the connection is made with a very large resistor of $1 G\Omega$. The lighter blue wire going into the "\~" port of the PMSM block is a three phase electrical cable, which is used throughout the SimPowerSystem sublibrary, and the Splitter collects three wires into a cable. \\

Current is sensed on wires A and B, and used to calculate $I_C$ in the upper part of the block diagram. The signal goes to a Clarke-Park transformation block outside figure~\ref{fig:simulations_electrical}, and returns as $I_d$ and $I_q$ on the bottom left part of the picture. \\

The angular position is measured with an ideal position sensor, and then sent to the encoder block. Here, the finite precision of the encoder is simulated by the equation~\ref{eq:Encoder_block_function}

\begin{equation}
output = floor \Bigg(\frac{\phi \cdot 256}{2 \pi} \% 256 \Bigg)
\label{eq:Encoder_block_function}
\end{equation}

where \% is the mod function. The purpose of that is to wrap the output to a value between 0 and 255, which can be used for look-up tables. The floor function rounds a number, effectively quantizing the output. It has been attempted to use the quantizer block, but that causes stiffness to the point where the simulations almost stall. The output is sent to the Zybo block, which will be explained in section~\ref{sub:sim_zybo}.\\

The Zybo generates duty cycles for each phase ranging from -1 to 1. This value is then multiplied with half the DC voltage, and saturated, o the ideal voltage sources do not provide more voltage than the battery can.

\subsubsection{Plecs model}\label{sub:sim_plecs_electrical}
As said, the Plecs model differs vastly from the Simulink model in the electrical network, as it more closely resemble the real analog circuit. It is presented on figure~\ref{fig:plecs_electrical}.

\begin{figure}[H]
	\begin{center}
		\includegraphics[width = \textwidth]{graphics/Plecs_electrical.pdf}
		\label{fig:plecs_electrical}
		\caption{Block diagram for the svm plecs simlations.}
	\end{center}
\end{figure}

\subsection{Zybo block}\label{sub:sim_zybo}