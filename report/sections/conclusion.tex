\section{Conclusion}
Throughout this report has been described the various steps taken to create a control system for an electric go-kart.
As part of this, circuit boards were developed to take care of over current protection, scaling of signals and various other tasks.
The OCP was simulated in order to verify its functionality
However, except for the driver board, due to time constraints it was not possible to finish the boards in time for the deadline.
A driver board was made to for the DRV8301, the driver chip used in the project.
This board was made to work with a test setup which includes the inverter developed, a small PMAC motor and some temporary boards made to interface the signals of the system.
Using this test setup it was possible to drive the motor, verifying that the inverter is functional.
A controller for the system was created using the settling time formulae.
The controller was tested using the extensive simulations made of the system using Simulink.
The controller was discretized in preparation for using it on the embedded system.
This was unfortunately not tested.
The embedded system was programmed using khaOS, a Run To Complete Scheduler.
Its real time performance was verified using an oscilloscope.\\

Many parts of the project were not completed to the satisfaction of the group.
Throughout there were problems with delivery of components as well as the quality of the boards created.
This resulted in much time spent doing things not necessarily conducive to a better end product.
