%!TEX root = ../main.tex
\newpage
\section{Controller Design}\label{sec:controller_design}
Controlling the PMAC motor will be done using the principles of field oriented control, which uses the Clarke Park Transform (dq0 transform) to transform three phase AC into two phase DC, allowing for easier regulation.
Controlling the PMAC motor is done by varying the PWM input to the inverter.
This is done by monitoring two of the three current inputs to the motor and transforming them using the dq0 transform as discussed in \ref{sec:clarkepark}. \newline

The requirement for this project outlines that the motor must be torque controlled, hence the flux demand will be zero while the torque demand will be defined by the torque pedal. 

The procedure of torque control is illustrated by figure \ref{fig:blockdiagram2}, which is repeated here for convenience.

\begin{figure}[H]
	\centering
	\includegraphics[width=1\linewidth]{graphics/ContolProcessDiagram}
	\caption{The control process as illustrated in the system overview section.}
	\label{fig:control_process_sec8}
\end{figure}

The three phase input to the motor is measured and transformed to dq.
After the transformation, the two signals, $I_d$ and $I_q$ are compared to the reference values and fed to the controllers.
The $I_d$ reference will be zero as this is the flux reference.
The $I_q$ reference is set by the torque pedal.
The controller acts upon these values ant they are transformed back to ABC three phase AC.
After this PWM is adjusted and the signals are routed through the driver to the inverter.

The torque reference, $I_q$, is in fact a current reference.
As such, torque control is equal to current control.
Taking this into account will be part of creating a functional model of the motor.\\

To model the PMAC motor, an equivalent circuit of the motor will be used. 
The equivalent circuit can be represented by the diagram presented on figure \ref{fig:PMSM_Equivalent_q_axis}.\newline

This figure shows that the q-axis of the PMAC motor can be represented as an electrical diagram with the physical properties represented by inductors and resistors.

\begin{figure}[!h]
	\centering
	\includegraphics[width=.4\textwidth]{graphics/q_axis_equivalent}
	\caption{PMAC motor equivalent circuit, q-axis.}
	\label{fig:PMSM_Equivalent_q_axis}
\end{figure}

The components shown are: the mutual inductance $l_s$ and $L_{mq}$. Armature resistance $R_s$ and flux linkage. 
The most important values here are the q-axis inductor and resistance which directly affect how much torque is gained from inputting current to the PMAC motor. 
During this section the components above will be named as follows.

\begin{equation}
R_{s}=R_a, \quad L_{mq}=L_a
\end{equation}

To current control the motor, the model and controller will be developed for this equivalent diagram. 
The q-axis by itself resembles the equivalent circuit of a DC motor and this can be used to design a mathematical model of the motor.

\subsection{State Space Model}
A state space model is derived from the physical representation of a PMAC motor, figure \ref{fig:State_space_model1}, consisting of an electrical part on the left and a mechanical part on the right side of the model. The model is made in MatLab Simulink. The model is very similar to that of a DC motor when interpreted like this.

\begin{figure}[H]
	\centering
	\includegraphics[width=.95\linewidth]{graphics/full_order_with_resistances_to_pdf}
	\caption{PMAC motor state space model.}
	\label{fig:State_space_model1}
\end{figure}

Reducing the model of figure \ref{fig:State_space_model1} to use transfer functions for the electrical and mechanical parts results in a more simple model with fewer loops. 
It is shown on figure \ref{fig:State_space_model3}. 
In the electric part of the circuit a new value $TS$ has been added to the denominator of the transfer function, this value is the electromagnetic time constant of the stator, defined as $TS=L_a/R_a$. The constant defined $J_{pro}$ is the projected inertia of the go-kart, based on the expected weight with a driver and also adding the inertia of each gear.

\begin{figure}[H]
	\centering
	\includegraphics[width=.95\linewidth]{graphics/full_order_as_tf_without_resistances_to_pdf}
	\caption{Transfer function PMAC motor model.}
	\label{fig:State_space_model3}
\end{figure}

\subsubsection{No Inductor Model}
The model from figure \ref{fig:State_space_model1} can be reduced by removing the inductance completely, this yields a model of reduced order. Reducing the order results in a model that can be easier to derive a controller for, since it removes a pole from the system. 
%The assumption here is that the inductor and integrators result in a very fast pole, that will be too fast to really affect the system. Whether this is a correct assumption will be tested with a simulation of either model. 
This reduced order model is seen on figure \ref{fig:State_space_model_reduced_order}.

\begin{figure}[H]
	\centering
	\includegraphics[width=.95\linewidth]{graphics/electrical_reduced_order_to_pdf}
	\caption{Reduced order PMAC motor model.}
	\label{fig:State_space_model_reduced_order}
\end{figure}

A pole-zero map of both models will be used to see whether the inductor can be removed without affecting the system significantly. Figure \ref{fig:PZmap_full} and \ref{fig:PZmap_Reduced} show this. 

\begin{figure}[H]
	\centering
	\begin{subfigure}[t]{.45\linewidth}
		\includegraphics[width=\textwidth]{graphics/full_order_model_pz_map}
		\caption{PZ map of full order model.}
		\label{fig:PZmap_full}
	\end{subfigure}
	\hspace{0.5cm}
	\begin{subfigure}[t]{.45\linewidth}
		\includegraphics[width=\textwidth]{graphics/reduced_order_model_pz_map}
		\caption{PZ map of no inductor model.}
		\label{fig:PZmap_Reduced}
	\end{subfigure}
	\caption{Matlab Simulink PZ Map of each model.}
\end{figure}

The pole-zero plot shows that the inductor pole is much faster than the mechanical pole, so it should be possible to remove it. \newline

The two state space models are simulated with MatLab Simulink to see how the system is affected by the removal of the inductor. 
A constant voltage of $26.2\si{\volt}$ is applied to each model.
Figure \ref{fig:Full_Vs_Reduced1_Current} and \ref{fig:Full_Vs_Reduced1_Speed} show these simulations.
\begin{figure}[H]
	\centering
	\begin{subfigure}[t]{.45\linewidth}
		\includegraphics[width=\textwidth]{graphics/Full_Vs_Reduced1_Current}
		\caption{Simulation of current for full and no inductor model.}
		\label{fig:Full_Vs_Reduced1_Current}
	\end{subfigure}
	\hspace{0.5cm}
	\begin{subfigure}[t]{.45\linewidth}
		\includegraphics[width=\textwidth]{graphics/Full_Vs_Reduced1_Speed}
		\caption{Simulation of speed for full and no inductor models (almost stacked on each other).}
		\label{fig:Full_Vs_Reduced1_Speed}
	\end{subfigure}
	\caption{Matlab simulation of each model with the same input.}
\end{figure}

 A problem shows up in these simulations. 
 The speed behaves well for the no inductor model, but the current starts at a high value. 
 The current starts out as a simple multiplication of the input voltage multiplied by the $1/R$ gain to equal $4000\si{\ampere}$. 
 The amplitude of speed and current are not important at this time, as that will be controlled later, but since the model does not ramp up the current like the  full order model, a controller will not be derived for this reduced order model.

\subsubsection{Reduced Order Model - No Mechanical Part}
\label{sec:reduced_model_1}
Because of the problem with reducing the model by removing the inductor, this section will focus on reducing the order by removing the mechanical part.
This model will be an experiment to remove the slow pole from the full order model. 
This will affect the system significantly. 
The mechanical system will now be seen as a disturbance. 
It can be seen on figure \ref{fig:State_space_model_reduced_order_v2}.

\begin{figure}[H]
	\centering
	\includegraphics[width=.55\linewidth]{graphics/mechanical_reduced_order_to_pdf}
	\caption{Reduced order PMAC motor model version.}
	\label{fig:State_space_model_reduced_order_v2}
\end{figure}

The model is basically an LR circuit where the output is the current. Which will just go as high as the L and R allows and stay there, since the back-EMF is not included. 
%A PZ map should shows that the slow pole to the right is removed. Figure \ref{fig:PZmap_reduced_v2} shows the result.
%\begin{figure}[H]
%	\centering
%	\includegraphics[width=.55\linewidth]{graphics/reduced_order_model_v2_pz_map}
%	\caption{PZ map of the reduced order model version 2}
%	\label{fig:PZmap_reduced_v2}
%\end{figure}

%The slow pole is gone, this is an important pole for the system, so removing it will be influential in the output. It  
%it is as slow as it is, so it might not be a good idea to remove it. The current must be controlled so leaving out all the m

% The removal of the pole requires the controller to be fast enough to react to subtle changes in the torque pedal, and robust because overshoot and steady state error should be intolerable, it should be able to handle all the mechanicals as a disturbance, but that might not be a possible outcome. 
Figure \ref{fig:Full_Vs_Reduced2} shows the current.

\begin{figure}[H]
	\centering
	\includegraphics[width=.55\linewidth]{graphics/Full_Vs_Reduced2_Current}
	\caption{Full order vs reduced order model.}
	\label{fig:Full_Vs_Reduced2}
\end{figure}

This shows that the current ramps up and stays there. The current starts at zero and ramps up. The downward slope of the full order model is caused by the back-EMF which is removed in the reduced model.
%
%This model could be used with a controller, but will require some simulations with the motor after a controller has been developed, the downward slope of the full order model is caused by the mechanical system getting saturated by current, this should not change how each model is affected by current control, because a fast and robust controller is desired.

\subsection{Controlling Current}
The motor must be current controlled in order to control the torque. 
%If the motor was speed controlled, as is often used in motor control, it could demand a much higher current than possible by the circuit, requiring a limiter on the current input instead, or in worst case it could result in destroying the drive circuit. A current control also means that
The torque pedal is used as the reference for the current controller, thus indirectly controlling the torque. 

A control loop is added to the model of figure \ref{fig:State_space_model3}. This means that a feedback loop is added from inside the motor model and fed back to the controller. On figure \ref{fig:State_space_model4} it is shown with the full order system.

\begin{figure}[H]
	\centering
	\includegraphics[width=.95\linewidth]{graphics/full_order_as_tf_with_controller_to_pdf}
	\caption{Full PMAC motor model with a current control loop.}
	\label{fig:State_space_model4}
\end{figure}

It would be preferable to have a second order controller for a second order system but the PI itself is preferred to avoid using a derivative in the discretized controller. A derivative term can become very inaccurate at high frequency. To improve precision it must be slowed down, but then the controller might be too slow all together. Hence if the derivative term can be avoided it will be. It should be able to control the 2nd order system based on the fact that one of the poles are very fast. The following sections will determine how well the PI controller can control the system.


\subsection{Controlling the Full Order System}
The model of figure \ref{fig:State_space_model4} can be rearranged to simplify the transfer function. This rearrangement can be seen on figure \ref{fig:State_space_model5}. Here the current is the output, a current reference is the input, and the mechanical part of the motor is all placed into a feedback loop. This reduces the complexity of the overall transfer function to only be a plant model with a PI controller added as shown in figure \ref{fig:State_space_model6}.

\begin{figure}[H]
	\centering
	\includegraphics[width=.95\linewidth]{graphics/full_order_rearranged_to_pdf}
	\caption{PMAC motor model rearranged to simplify the transfer function.}
	\label{fig:State_space_model5}
\end{figure}

\begin{figure}[H]
	\centering
	\includegraphics[width=.95\linewidth]{graphics/plant_with_pi_to_pdf}
	\caption{PMAC motor model as a plant with PI control.}
	\label{fig:State_space_model6}
\end{figure}

Using 'Masons Rule'\cite{feedback}, the transfer function can be derived. The following equation is achieved:

\begin{equation}
\dfrac{I_q}{I_{q_{ref}}}= \dfrac{(K_p+K_i\frac{1}{s})G(s)}{1+G(s)(K_p+K_i \frac{1}{s})}
\label{eq:plant_masons}
\end{equation}

$G(s)$ from figure \ref{fig:State_space_model1} is:

\begin{equation}
G(s)=\frac{2 J_{pro} s + 2 K_v}{s^2(2 J_{pro} L_a) + s(2 K_v L_a + 2 J_{pro} R_a)+3 K_t^2 + 2K_v R_a}
\label{eq:full_order_tf}
\end{equation}

%Analysing this transfer function it can be seen that we have a second order system, with possibly one zero and one or two poles. The zero will be located at the root of the nominator: $2J_{pro}s+2K_v=0 \rightarrow s=\frac{-K_v}{J_{pro}}$. Which will probably be a very small real number, as inertia should be much larger than the viscous friction. 
%The pole(s) will be located at the roots of the denominator.
 The denominator can be reduced by dividing everything with $2J_{pro}L_a$. yielding the transfer function in equation \ref{eq:big_guy_huh}.

\begin{equation}
G(s)=\frac{\frac{1}{L_a}s+\frac{K_v}{J_{pro}}}{
s^2+s \left( \frac{K_v}{J_{pro}}+\frac{R_a}{L_a} \right) +\frac{3}{2}\frac{K_t^2}{J_{pro}L_a}+\frac{K_vR_a}{J_{pro}L_a}}
\label{eq:big_guy_huh}
\end{equation}

%Based on the form of the denominator being a second order equation, there could be complex conjugate poles. This cannot be known until all values are inserted and the roots of the denominator found. This means that it could have any number of characteristic. The pole-zero map with the values included was used in figure \ref{fig:PZmap_full}.
% Two poles on the real axis and a zero close to the imaginary axes mean that the system is stable with decay. \newline

Inserting \ref{eq:big_guy_huh} in \ref{eq:plant_masons} and rearranging to get the form of a characteristics equation the transfer function is equation \ref{eq:transfer_function_full}.

%old equation. please leave (Morten)
\begin{equation}
\frac{I_q}{I_{q_{ref}}}=\frac{s^2\frac{K_p}{L_a}+s(\frac{K_i}{L_a}+\frac{K_pK_v}{J_{pro}})+\frac{K_iK_v}{J_{pro}}}{s^3+s^2(\frac{R_a}{2L_a}+\frac{K_v}{J_{pro}}+\frac{K_p}{L_a})+s(\frac{3K_t^2}{2J_{pro}L_a}+\frac{K_vR_a}{J_{pro}L_a}+\frac{K_i}{L_a}+\frac{K_pK_v}{J_{pro}L_a})+\frac{K_iK_v}{J_{pro}L_a}}
\label{eq:transfer_function_full}
\end{equation}


%2nd order transfer function: dont delete
%\begin{equation}
%\dfrac{I_m}{I_{ref}}=\dfrac{L_a(K_i+K_ps)}{s^2+s(\frac{R_a}{L_a}+\frac{1}{L_a}K_p)+\frac{1}{L_a}K_i+\frac{3K_t^2}{2JL_a}}
%\label{eq:transfer_function_full2}
%\end{equation}

This third order transfer function will be used to derive the controller values.

\subsection{Full Order Controller Values}\label{sub:full_order_controller_values}
To find the controller values the Stephen Dodds Settling Time Formula for $5\%$ settling time is used\cite{feedback}, The desired characteristics polynomial of a transfer function for the settling time formula to work is equation \ref{eq:char_eq_wanted}.

\begin{equation}
s^n+d_{n-1}s^{n-1}+\cdot \cdot \cdot +d_1s+d_0
\label{eq:char_eq_wanted}
\end{equation}

The d's of a given order can be derived by solving $(s+d)^n$ 
for the given order of the characteristics equation.
This yields the following third order equation for a third order system.

\begin{equation}
s^3+3\alpha s^2+3\alpha^2s+\alpha^3
\label{eq:s_form_equation}
\end{equation}

Based on the transfer function in equation \ref{eq:transfer_function_full} and a settling time of 0.05 seconds, $\alpha$ will be defined by the following equation:

\begin{equation}
\alpha=\frac{1.5(1+n)}{T_s^{5\%}}=\frac{6}{0.05}=120
\label{eq:Dodds_settlingtime}
\end{equation}

Characteristics equation of \ref{eq:full_order_tf} is equation \ref{eq:big_guy_with_cont}.

\begin{equation}
s^3+s^2\left(\frac{R_a}{2L_a}+\frac{K_v}{J_{pro}}+\frac{K_p}{L_a}\right)+s\left( \frac{3K_t^2}{2J_{pro}L_a}+\frac{K_vR}{J_{pro}L_a}+\frac{K_i}{L_a}+\frac{K_pK_v}{J_{pro}L_a}\right)+\frac{K_iK_v}{J_{pro}L_a}
\label{eq:big_guy_with_cont}
\end{equation}

Grouping constants yields the following form:

\begin{equation}
s^3+s^2(a_2+b_2K_p)+s(a_1+b_1K_i)+a_0+b_0
\end{equation}

Setting this equation equal to \ref{eq:s_form_equation} yields:

%%%Morten: These calculations are for the reduced order model (v1). please leave until complete.
%\begin{align}
%a_2&=\frac{K_v}{J_m}+\frac{R_a}{L_a}  &b_2&=\frac{1}{L_a}\nonumber\\
%a_1&=\frac{K_vR_a}{J_mL_a}+\frac{3}{2}\frac{K_t^2}{J_mL_a} &b_1&=\frac{1}{L_a}+\frac{K_pK_v}{J_mL_a}\nonumber\\
%a_ 0&=0  &b_0&=\frac{K_v}{J_mL_a}\nonumber
%\end{align}




\begin{equation}
s^3+s^2(a_2+b_2K_p)+s(a_1+b_1K_i)+a_0+b_0 = s^3+3\alpha s^2+3\alpha^2s+\alpha^3
\label{eq:doddsnormaleq}
\end{equation}

Solving the above equation for $K_p$ and $K_i$ yields the equation two equations \ref{eq:solved_d1} and \ref{eq:solved_d2}.
%Old calculations of kp=0.0039 and Ki=0.4606. please leave it.
%\begin{equation}
%3 \alpha= \frac{R_a}{2L_a}+ \frac{K_v}{J_{pro}}+ \frac{K_p}{L_a}
%\label{eq:solved_d1}
%\end{equation}
\begin{equation}
K_p=(3 \alpha - \frac{K_v}{J_{pro}}- \frac{R_a}{2L_a})L_a
\label{eq:solved_d1}
\end{equation}\newline
%\begin{equation}
%3 \alpha^2= \frac{2K_t^2}{2J_{pro}L_a}+ \frac{K_vR_a}{J_{pro}L_a}+ \frac{K_i}{L_a}+ \frac{K_pK_v}{2J_{pro}}
%\label{eq:solved_d2}
%\end{equation}
\begin{equation}
K_i=(3 \alpha^2- \frac{K_pK_v}{2J_{pro}}- \frac{K_vR}{J_{pro}L_a}- \frac{3K_t^2}{2J_{pro}L_a})L_a
\label{eq:solved_d2}
\end{equation}


%%Morten: Calculations from reduced order transfer function that does not work.
%From equation \ref{eq:doddsnormaleq}, the following equations can be made:
%\begin{equation}
%a_2+b_2K_p=d_2 \rightarrow K_p\frac{d_2-a_2}{b_2}
%\end{equation}
%
%\begin{equation}
%a_1+b_1K_i=d_1 \rightarrow K_i\frac{d_1-a_1}{b_1}
%\end{equation}

These equations yield the following controller values when inserting the known motor parameters:

\begin{equation}
K_p=0.0111
\label{eq:Kpvalue}
\end{equation}
\begin{equation}
K_i=1.7566
\label{eq:Kivalue}
\end{equation}

The pole-zero map for the transfer function including the controller confirms that a new pole and zero has been added. All poles and zeroes lie on the real axis, meaning they affect the system by altering the decay while no oscillation is introduced. The system is stable with these values since they are all on the left side of the imaginary axis. See figure \ref{fig:PZmap_Control_full}. How the controller affects the output will be tested further on.

\begin{figure}[H]
	\centering
	\includegraphics[width=.55\linewidth]{graphics/full_order_w_pid_to_pdf}
	\caption{PZ map with the controller.}
	\label{fig:PZmap_Control_full}
\end{figure}


\subsection{Controller for the Reduced Transfer Function}
In section \ref{sec:reduced_model_1} a reduced order model of the PMAC motor was derived. This model removed the mechanical part of the model to reduce the order. This section will develop a controller for this model.

The same procedure as for the full order system will be used for this transfer function. The open loop transfer function is equation \ref{eq:small_trans}.

\begin{equation}
\frac{I_q}{V_{in}}=H(s)=\frac{1}{R_a+L_as}
\label{eq:small_trans}
\end{equation}

%There are no zeros as the numerator has no s values and there is one real pole at $R_a+L_as=0 \rightarrow s=0$. Which equals to (0,0) in the pole-zero map. This means that as the gain increases this pole becomes faster and faster (moves to the left). With the pole right on the imaginary axis the transfer function itself is marginally damped, and will be stable by itself. It is basically a constant value determined by the initial conditions. The controller is still needed to set a reference value though. With no mechanical part there is no decay at all, but a test will have to be performed to see how well a controller can react to the disturbance of the mechanical part.

\subsection{Reduced Order Controller Values}\label{sub:reduced_order_controller_values}
Finding the controller values for this model is done as it was for the full order model. However an IP controller will be used instead to avoid adding a zero to the transfer function.

 The IP controller places the $K_p$ value in a feedback loop resulting in the following diagram on figure \ref{fig:IPcontroller1}.

\begin{figure}[H]
	\centering
	\includegraphics[width=.95\linewidth]{graphics/plant_with_ip_to_pdf}
	\caption{Reduced order system with an IP controller.}
	\label{fig:IPcontroller1}
\end{figure}

The closed loop transfer function of figure \ref{fig:IPcontroller1} is:
\begin{equation}
\frac{I_q}{I_{qref}}=\frac{\frac{K_t}{L_a}}{s^2+s(\frac{R_a}{L_a}+\frac{K_p}{L_a})+\frac{K_t}{L_a}}
\label{eq:trans_reduced1}
\end{equation}

Using Dodd's $5\%$ settling time formula\cite{feedback} and solving for a 2nd order system and a settling time of $0.05s$. The results are equation \ref{eq:sett_time_eq}
\begin{equation}
\alpha=\frac{1.5(1+2)}{0.05}=90
\label{eq:sett_time_eq}
\end{equation}

The desired characteristics equation has the form of the following equation (\ref{eq:char_eq_reduced1}).
\begin{equation}
s^2+sd_1+d_0 = s^2+2 \alpha s+ \alpha^2
\label{eq:char_eq_reduced1}
\end{equation}

Solving for $K_p$ and $K_p$ and inserting the known motor parameters yields the following controller values:
\begin{equation}
2 \alpha =\frac{R_a}{L_a}+\frac{K_p}{L_a} \rightarrow K_p=7e^{-4}
\label{eq:Kp_redux}
\end{equation}
\begin{equation}
\alpha^2=\frac{K_i}{L_a} \rightarrow K_i=0.324
\label{eq:Ki_redux}
\end{equation}

The closed loop transfer function results in the pzmap of figure \ref{fig:PZmapIPcontroller1}. Since there are no zero in the transfer function, there wont be one when adding an IP controller. The added pole becomes a complex conjugate pair. This means that the system is stable but might have oscillation. This will be tested in a simulation in the next section.

\begin{figure}[H]
	\centering
	\includegraphics[width=.55\linewidth]{graphics/reduced_order_model_w_ip_controller_to_pdf}
	\caption{PZ map of the reduced order system with an IP controller.}
	\label{fig:PZmapIPcontroller1}
\end{figure}


\subsection{Control Simulation}
The derivations above have resulted in controller gain values as seen in table \ref{tab:cont_gains}.

\begin{table}[H]
\centering
\begin{tabular}{lll}
\cline{1-3}\rowcolor[HTML]{C0C0C0}
\multicolumn{1}{|l|}{Model/Gain} & \multicolumn{1}{l|}{$K_p$} & \multicolumn{1}{l|}{$K_i$}     \\ \cline{1-3}

\multicolumn{1}{|l|}{Full Order} & \multicolumn{1}{l|}{0.0111} & \multicolumn{1}{l|}{1.7566}    \\ \cline{1-3}

\multicolumn{1}{|l|}{Reduced Order} & \multicolumn{1}{l|}{$7e^{-4}$} & \multicolumn{1}{l|}{0.324}    \\ \cline{1-3}

\end{tabular}
\label{tab:cont_gains}
\caption{$K_p$ and $K_i$ derived in previous sections.}
\end{table}


The controllers should produce a stable current output from the electrical part of the motor, which is desired to control the torque.

Figure \ref{fig:Input_ref} is the input reference signal which simulates the torque pedal being fully actuated, then held for a few seconds and then released over two seconds. Full actuation is set to $300\si{\ampere}$ and the run time is set to 10\si{\second}.

\begin{figure}[H]
	\centering
	\includegraphics[width=.50\linewidth]{graphics/Control_signal_input}
	\caption{Input reference for the controller.}
	\label{fig:Input_ref}
\end{figure}

Figure \ref{fig:output_both} shows how both models perform with the controller values found above and the input signal. The full order model has a small steady state error of slightly less than $5A$ while the reduced order model is spot on. The settling time for both is higher than the $0.05$ value set for the settling time formula though. The full order model settles in $0.9s$ while the reduced model settles is $0.13s$. They both handle the control as they should, with no overshoot at the start, or undershoot at the end of the breaking.

\begin{figure}[H]
	\centering
	\begin{subfigure}[t]{.45\linewidth}
		\includegraphics[width=\textwidth]{graphics/Controlled_response_Fullorder}
		\caption{Current output response of the full order model to the signal input given in figure \ref{fig:Input_ref}.}
		\label{fig:Output_Full}
	\end{subfigure}
	\hspace{0.5cm}
	\begin{subfigure}[t]{.45\linewidth}
		\includegraphics[width=\textwidth]{graphics/Controlled_response_LR}
		\caption{Current output response of LC model to the signal input given in figure \ref{fig:Input_ref}.}
		\label{fig:Output_LC}
	\end{subfigure}
	\caption{Simulations of each model with their controllers.}
	\label{fig:output_both}
\end{figure}

The controller to be used for the go-kart will be determined by running a more in depth simulation in the next section, this simulation will include the physical systems, breaks, torque etc. A choice will be made based on controller performance in these tests.


\subsection{Discretization}
Discretizing the controller is required to use it in the software. 
To do this, the values of $K_p$ and $K_i$ can be used as they are, but for both a PI and IP controller the integrator must be discretized. 
An integrator in discrete is done by using a numerical integration method. The choice here will be the Trapezoidal method, which has the output equation given as equation \ref{eq:trap_integrat}. 

\begin{equation}
y_{n+1} = y_{n} + \frac{1}{2}h \left( f(t_n,y_n)+f(t_{n+1},y_{n+1}) \right)
\label{eq:trap_integrat}
\end{equation}

Here n is the time step, t is the time and y is the function to integrate. The trapezoidal method yields a more precise method than forward or backwards Euler methods, due to it being 2nd order precise ($O(h^2)$), but it takes twice the computation power.

\subsection{Third Harmonic Injection}\label{sub:third_harmonic_injection}
Due to the limited battery voltage, it is not possible to drive the motor to speeds close to the maximum set by the manufacturer.
Because the internal star point of the motor is not connected to the outside, it is possible to gain an additional 15\% by way of third harmonic injection.
The reason for this is, that the controller can add the same voltage offset to all three terminals of the motor without changing the currents, thus avoiding the peaks of the sinusoidals.
There are several ways of doing this, but only the sinusoidal will be considered here. 
Adding a $\frac{1}{6}$ of the third harmonic will result in 15 \% more line to line voltage. 
The resulting waveforms can be seen in section~\ref{fig:Sim_results_pi_phase_voltage}, on page~\pageref{fig:Sim_results_pi_phase_voltage}.\\

The issue is that, when using field oriented control, the stator voltage does not remain at a constant angle relative to the rotor angle.
That means, that calculating the third harmonic isn't as simple as multiplying the rotor position with three. 
Doing this will result in the the third harmonic being out of phase with the three sinusoidal voltage waveforms as speed increases.
This will result in higher peak voltages at certain speeds.
Instead, the third harmonic needs to be calculated from the dq output from the controller. \\

Equation (9) in the paper\cite{third_harmonic_injection} defines the three phase voltage of a balances system without third harmonic injection:

\begin{equation}
\left[ \begin{array}{c}
V_A \\ V_B \\V_C
\end{array} \right]
=
\left[ \begin{array}{c}
V \sin(\Theta) \\
V \sin(\Theta - 120 \si{\degree}) \\
V \sin(\Theta + 120 \si{\degree})
\end{array} \right]
\label{eq:equation_9}
\end{equation}

where $\mathrm{\Theta}$ is the instantaneous phase, and V is the instantaneous magnitude, calculated as:

\begin{equation}
V = \sqrt{V_q ^2 + V_d ^2}
\label{eq:v_from_vdq}
\end{equation}

Note that equation~\ref{eq:v_from_vdq} is not equal to equation 10 in the paper\cite{third_harmonic_injection}, because this dividing with 1.5 is done in the Clarke Park transform block. 

From equation~\ref{eq:equation_9}, we know that:

\begin{equation}
\Theta = \sin^{-1} \Big(\frac{V_A}{V}\Big)
\label{eq:equation_11}
\end{equation}

The third harmonic sinusoidal is given in equation (12) of the paper\cite{third_harmonic_injection}, and shown in equation~\ref{eq:v_th}

\begin{equation}
V_{th} = \frac{1}{6} \sin(3 \Theta) = \frac{1}{6} \sin\Bigg(3 \sin^{-1} \Big(\frac{V_A}{V}\Big)\Bigg)
\label{eq:v_th} 
\end{equation}

This third harmonic sinusoidal is added to each of the phases, as shown in equation~\ref{eq:equations_14}

\begin{equation}
\begin{bmatrix}
V_{Ath} \\
V_{Bth} \\
V_{Cth}
\end{bmatrix}
=
\begin{bmatrix}
V_A + V_{th} \\
V_B + V_{th} \\
V_C + V_{th}
\end{bmatrix}
\label{eq:equations_14}
\end{equation}

\clearpage