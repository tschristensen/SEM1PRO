
\section{Three phase inverter}



\subsection{Choosing the MOSFETs}

Before choosing the MOSFETs one will have to keep in mind the application parameters.
Firstly the battery voltage of the Super-B(SB12V20P-FC)\todo{Erlingur - Super B datasheet} will on full charge have nominal voltage of 52.8V and a maximum end of charge voltage at 60.8V. 
The MOSFETs will therefor need to be able to handle a minimum drain to source voltage, $V_{DSS}$, of 60.8V. 
Secondly, the current drawn by the motor can reach a maximum peak of 300A.
Thirdly, the motor driver(DRV8301)\todo{Erlingur - DRV8301 datasheet} will supply a gate voltage, $V_{GS}$, of 9.5V to 11.5V. 
The MOSFETs will therefor need to be fully operating at $V_{GS} = 9.5V$ and be able to handle the driver supplied range.
The three main requirements are listed in table \ref{tab:req1}\\

\begin{table}[h!]
	\caption{Three main requirements as a starting point for choosing the MOSFETs}
	\centering
	\begin{tabular}{clr}
		  & Requirement\\
		  \hline
		1 & Can handle minimum voltage of & 60.8V\\
		2 & Can handle minimum current of & 300A\\
		3 & Operating at $V_{GS}$ of  & 9.5V \\
		\hline
	\end{tabular}
	\label{tab:req1}
\end{table}

Do to high currents through the MOSFETs, the heat dissipation is assumed to be high as well. 
To prevent overheating and burning out the MOSFETs they will be mounted on to a heat sink. 
Therefore a through hole package will be suitable for this application. The chosen package is the TO-247. And the MOSFET IRFP4468PBF is chosen\todo{Erlingur - why this package and MOSFET?}.

\begin{table}[h!]
	\centering
	\caption{Some important parameters from the IRFP4468PBF MOSFET datasheet... missing citation. all parameters are found at junction temperature $T_J = 25^{\circ}$C}
\begin{tabular}{|l|l|c|c|c|c|}
	\hline
	\multicolumn{1}{|c|}{Symbol} & \multicolumn{1}{|c|}{Parameter} & Min. & Typ. &
	 Max. & Units\\
	\hline
	$I_D$ & Continuous Drain Current (Silicon Limited) & - & - & 290 & A\\
	\hline
	$I_D$ & Continuous Drain Current (Package Limited) & - & - & 195 & A\\
	\hline
	$V_{GS}$ & Gate to Source Voltage & - & - & $\pm$20 & V\\
	\hline
	 $V_{GS(th)}$ & Gate Threshold Voltage & 2.0 & - & 4.0 & V\\
	 \hline
	 $V_{(BR}DSS$ & Drain to Source Breakdown Voltage & 100 & - & - & V\\
	 \hline
	 $R_{DS(on)}$ & Static Drain to Source On Resistance & - & 2.0 & 2.6 & m$\Omega$\\
	 \hline
	 
\end{tabular}
\label{tab:MOSdata}
\end{table}

The MOSFET parameters in table~\ref{tab:MOSdata} seems to add up to all the requirements in table~\ref{tab:req1}. 
However the TO-247 package has a continuous drain current limit of 195A that is only around two third of the minimum current requirement. 
This problem will be solved using two MOSFETs in parallel. 
Using the parallel setup, the two MOSFETs will share the current equally so that each will have a maximum peak current of 150A.
As will be later discussed, the setup will also decrease the conduction losses but increase the switching losses\todo{Erlingur - How will it increase the switching loss}.


\subsection{Power Losses}

To estimate the temperature, the power loss must be estimated. 
The total power loss, $P_{loss}$ of a switching device can be estimated using equation~\ref{eq:p_loss} where $P_C$ and $P_{SW}$ are conduction- and switching losses respectively.  

\begin{equation}
P_{loss} = P_{C} + P_{SW}
\label{eq:p_loss}
\end{equation}

Conduction loss of a single MOSFET can be calculated using equation~\ref{eq:p_c} and the switching loss with equation~\ref{eq:P_sw}

\begin{equation}
P_{C} = R_{DS(on)} I_{D,rms}^2
\label{eq:p_c}
\end{equation}

\begin{equation}
P_{SW} = \dfrac{1}{2} V I t f_{SW}
\label{eq:P_sw}
\end{equation}

For simplicity the power loss will be estimated from a single phase of the inverter circuit seen in figure~\ref{fig:single_phase}. 
The Current $I_{out}$ is drawn by some load $R_{load}$ that will absorb the load power $P_{load}$ which is given by equation\ref{eq:p_load}. 
Since there are two switches, the load power must be delivered equally from both. Thus the output power, $P_{Q}$, of each switch is given by equation~\ref{eq:P_Q}  



\begin{figure}[h!]
	\centering
	\includegraphics[scale = 0.6]{graphics/single_phase}
	\caption{Single phase}
	\label{fig:single_phase}
\end{figure}


\begin{equation}
P_{load} = R_{load} I_{out,rms}^2
\label{eq:p_load}
\end{equation}

\begin{equation}
P_{Q} = \dfrac{P_{load}}{2} = R_{load} I_{Q,rms}^2
\label{eq:P_Q}
\end{equation}

With insertion of equation~\ref{eq:p_load} into~\ref*{eq:P_Q} and solving for $I_{Q,rms}$, equation~\ref{eq:I_qrms} is formed.

\begin{equation}
\dfrac{R_{load} I_{out,rms}^2}{2} = R_{load} I_{Q,rms}^2 \rightarrow I_{Q,rms} = \dfrac{I_{out,rms}}{\sqrt{2}} 
\label{eq:I_qrms}
\end{equation}
Which then again can be expressed as

\begin{equation}
I_{Q,rms} = \dfrac{I_{out,rms}}{\sqrt{2}} = \dfrac{I_{out}}{\sqrt{2}\sqrt{2}} = \dfrac{I_{out}}{2}
\end{equation}
As mentioned earlier each switch will consist of two MOSFETs. Therefor the Drain current of each MOSFET becomes

\begin{equation}
I_{Qi,rms}=\dfrac{I_{Q,rms}}{2} = \dfrac{I_{out}}{4}
\label{I_Qi}
\end{equation}

Although listed in the datasheet, the drain to source resistance $R_{DS(on)}$ only applies to the  $T_J = 25^{\circ}$C condition. 
The resistance will vary with the junction temperature $T_J$. 
The relation between $T_J$ and $R_{DS(on)}$ can be seen in figure~\ref{fig:tj_rds} from where the multiplication factor can be estimated\todo{Erlingur-put the eq in}.

\begin{figure}[h!]
	\centering
	\includegraphics[scale =0.7]{graphics/T_n_R_compare}
	\caption{The $R_{DS(on)}$ and $T_J$ relation at $I_D = 180A$ and $V_{GS} = 10V$}
	\label{fig:tj_rds}
\end{figure}

If assumed that the junction temperature will be around $100^{\circ}C$ one can estimate the factor being somewhere close to 1.6. 
Then $R_{DS(on),@100^{\circ}C}$ will have the value

\begin{equation*}
R_{DS(on),@100^{\circ}C} =  1.6 R_{DS(on),@25^{\circ}C}
\end{equation*}

then the conduction loss can be estimated using~\ref{eq:p_c} and \ref{I_Qi}

\begin{equation*}
P_{C} = R_{DS(on),@100^{\circ}C} I_{Qi,rms}^2
\end{equation*}

\subsection{Switching loss}
The switching loss happens when the MOSFETs go from the on state to the off state and vice versa. This loss can be found with experiments. 
When determining the switching loss only from calculations, some assumptions have to be made.
We assume that the gate to source capacitance $C_{GS}$ is the same for on and off time. 
The gate to source current is defined by equation~\ref{I_gs}. 


\begin{equation}
I = C\dfrac{dV}{dt}
\label{I_gs}
\end{equation}

However, the gate to source current is known. the driver supports up to 2.3A sink and 1.7A source current. 

\begin{figure}
	\centering
	\includegraphics[scale = 0.7]{graphics/sw_time_wavef}
	\caption{Switching Time Waveforms}
	\label{sw_time_wavef}
\end{figure}

\begin{table}
	\centering
	\begin{tabular}{|c|c|c|c|}
		\hline
		Symbol & Parameter & Typ. & Units\\
		\hline
		$t_r$ & Rise Time & 230 & ns\\
		$t_f$ & Fall Time & 260 & ns\\
		\hline
	\end{tabular}
\end{table}

\pagebreak