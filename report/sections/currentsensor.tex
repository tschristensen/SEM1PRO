\section{Current Transducer - LF 205-S}
The LF 205-S utilises the Hall effect in order to measure the current flowing through a wire.
Figure \ref{fig:lf205function} depicts an equivalent circuit of the functionality of the transducer.
As can be seen, the LF 205-S functions like a transformer.
As per the datasheet this transformer has a 1:2000 turns ratio, therefore:
\begin{equation}
	\frac{I_S}{I_P}=\frac{N_P}{N_S} \quad \Rightarrow \quad I_S = \frac{I_P}{N_S}
\end{equation}
Essentially, the LF 205-S generates a current, $I_S$, proportional to the current flowing through the device, $I_P$.
By changing the value of the resistor $R_m$, the resulting voltage drop can be dictated.

In section \todo{Section documenting the maximum current to be expected in each phase.} the maximum current to be expected in each phase is found to be 115 A.
This value can be both positive and negative.
In order to properly read 