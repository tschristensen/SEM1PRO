\section{Current Transducer - LF 205-S}
The LF 205-S utilises the Hall effect in order to measure the current flowing through a wire.
Figure \ref{fig:lf205function} depicts an equivalent circuit of the functionality of the transducer.
As can be seen, the LF 205-S functions like a transformer.
As per the datasheet this transformer has a 1:2000 turns ratio, therefore the secondary current, $I_S$ can be found as:
\begin{equation}
	\frac{I_S}{I_P}=\frac{N_P}{N_S} \quad \Rightarrow \quad I_S = \frac{I_P}{N_S}
\end{equation}
Essentially, the LF 205-S generates a current proportional to the current flowing through the device, the primary current; $I_P$.
By changing the value of the resistor $R_m$, the resulting voltage drop can be dictated.

In section \todo[inline]{Thomas: Section documenting the maximum current to be expected in each phase.} the maximum current to be expected in each phase is found to be 300$A$.
As described in section \todo[inline]{Thomas: section on the zybo adc}, the ADC on the Zynq chip can work on signals in the ranges 0$V$-1$V$ or $\pm0.5V$.
Since the current is sinusoidal it can be negative.
Choosing the range $\pm0.5V$ therefore simplifies the circuitry needed to read the value correctly.
Consequently $R_m$ needs to be set such that the maximum expected current results in the maximum allowed voltage.
However, it is desirable to leave some headroom in case of an overcurrent event, 50$mV$ should be sufficient.
Thus:
\begin{equation}
	R_m = \frac{0.45}{I_{Pmax}/N_S} = 3\Omega
\end{equation}
  