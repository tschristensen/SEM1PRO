%!TEX root = ../main.tex

\section{System Analysis}
To get an overview of the system connections and all the major components needed to make the kart run, a system diagram is made based on the wiring diagram seen in figure \ref{fig:Kart_wiring_diagram}. 
Figure \ref{fig:blockdiagram1} shows a simplified diagram of the wiring, where the Zybo board has replaced the Sevcon Gen4 controller. \newline

The highlighted green component is the Zybo board.
The PMAC motor, the driver and the inverter are all placed to the right of the controller board.
Both the driver and inverter are parts of the Sevcon Gen4 controller and as such, they must be developed.
The two blocks next to the motor are LEM current sensors. 
Two of these are sufficient as the third current can be calculated using Kirchoff's current law.
The encoder is measuring the rotational position of the rotor and feeding it to the controller.
All three encoder signals might not be needed but they will be routed to have the opportunity to do so. \newline

To the left of the Zybo the different safety systems are placed.
These will ensure that starting, stopping and emergency procedures are handled safely, preferably without any burned components.
On top are the different switches, the key switch, the emergency switch and the enable switch.
Below these is the undervoltage protection circuit which ensures that the kart is only driving when sufficient power is supplied. 
The final parts to the left are the torque pedal for the kart and the battery supply.

\begin{figure}[H]
	\centering
	\includegraphics[width=.85\linewidth, trim=0cm 4cm 0cm 3cm]{graphics/wiringdiagram}
	\caption{Wiring diagram with the Zybo board.}
	\label{fig:blockdiagram1}
\end{figure}

A diagram of the functionality of the wanted control process can be seen in figure \ref{fig:blockdiagram2}.
The different blocks of the figure will be explained in depth in the coming sections, but is presented here to give an overview of the complete system.

\begin{figure}[H]
	\centering
	\includegraphics[width=1\linewidth]{graphics/ContolProcessDiagram}
	\caption{Diagram of the drive control system}
	\label{fig:blockdiagram2}
\end{figure}

The following sections will explain what needs to be done in different areas to achieve the functionality of figure \ref{fig:blockdiagram1} and \ref{fig:blockdiagram2}    

\subsection{Electronics}
In order to fulfill the task some electric circuits will be developed to perform certain tasks. 
The tasks at hand for the electronic circuits were found to be: 

\begin{itemize}
\item Overcurrent protection.
\item Circuitry for LEM current transducer output.
\item Power supplies Zybo board, current transducers and other parts of the circuit.
\item Torque pedal downscale. The torque pedal output must be scaled as to not burn the ADC on the Zybo board.
\item Various logic to control different parts of the system, including logic level shifting for the encoders as well as circuitry for a drive enable signal.
\item Driver circuit. A driver is needed to control the inverter.
\end{itemize}
All of the above points will be explored further in section \ref{sec:electronics}.

\subsection{Embedded System}
An embedded system is needed in the project in order to perform certain tasks.
These tasks include:

\begin{itemize}
\item Creation of PWM signals. The motor will require a three phase sinusoidal signal in order to run, this is created using PWM.
\item Controller. The digital controller used to control the entire system will be implemented using the embedded system.
\item Interface for peripherals. A number of switches, LED's, enable and fault signals are used in the control of the motor.
\item Conversion of the analog signal created by the current transducers and the torque pedal into a digital signal which can be used in controlling the motor.
\item Decoding the input from the encoder.
\item Interface to a computer in order to set and get relevant parameters.
\end{itemize}
The overall software design and development will be explored further in section \ref{sec:emb}.

\subsection{Power Electronics}
A three-phase inverter will be developed throughout this project.
The functionality of the inverter is to convert the DC voltage of the batteries into a three-phase sinusoidal signal needed to run the motor.
Much of the complication in this part lies in choosing the right components for the high currents necessary for driving the motor.
The development of the power electronics is discussed in detail in section \ref{sec:inverter}.
This includes aspects, such as calculating switching losses and heat dissipation.

\subsection{PMAC Motor}
The PMAC motor provided needs to be analysed in order to drive it optimally and simulate the performance of it. 
Parameterisation of the motor will be done as a part of the analysis. 
This analysis as well as a general explanation of the functionality of a PMAC motor will be presented throughout section \ref{sec:PMAC}.

\subsection{Control Theory}
In order to make the go kart drive optimally a controller needs to be developed. 
A three phase PMAC motor needs a three phase sinosoidal signal to drive and as this is problematic to control directly some transformation will be needed. 
As the control algorihm will need to be run on a embedded system a discretization of the controller will be needed. 
The development of the controlelr will be described in detail in section \ref{sec:controller_design}.

\clearpage