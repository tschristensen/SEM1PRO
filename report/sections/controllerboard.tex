\section{Controller Board}
This section describes the different parts present on the board housing the motor driver. 
DRV8301DCA

\subsection{Logic-Level-Converters}
\label{sec:llc}
The encoder present on the motor uses 5V signals for communication while the Z-7010 FPGA present on the Zybo board is a 3.3V chip. 
For this reason it is necessary to shift the logic levels between these chips to ensure compatibility.
This is done using simple logic level converters (LLCs).
Five signals require shifting in total, four of which are positional information about the motor sent to the FPGA.
These four signals therefore need to be downscaled such that the voltage seen by the FPGA is $3.3\pm0.2$V\cite{z7010}.
This can be done by a simple voltage divider as seen in figure \ref{fig:llc5to33}.
To ensure that the output and input resistances ($R_{out}$, $R_{in}$) of the encoder and the FPGA, respectively, do not influence the circuit, it is necessary that $R_{out}\ll R_1\ll R_{in}$.
Usually, \todo{perhaps actually find the output and input resistances?} $R_{out}<100\Omega$ and $R_{in}>500k\Omega$.
Then the ratio of the resistances $R_1$ and $R_2$ is:
\begin{equation}
	\frac{V_{out}}{V_{in}} = \frac{R_2}{R_1+R_2} = \frac{3.3}{5}
\end{equation}
Thus $R_2 = 33k\Omega$ and $R_1 = 17k\Omega$ satisfies the requirement.

\begin{figure}[!h]
	\begin{subfigure}[t]{.49\linewidth}
			\centering
			\includegraphics[width=\textwidth]{graphics/llc_5_33}
			\caption{Logic Level Converter: $5V\rightarrow3.3V$.}
			\label{fig:llc5to33}
	\end{subfigure}
	\begin{subfigure}[t]{.49\linewidth}
		\includegraphics[width=\textwidth]{graphics/llc_33_5}
		\caption{Logic Level Converter: $3.3V\rightarrow5V$.}
		\label{fig:llc33to5}
	\end{subfigure}
	\caption{Logic level converters used to ensure compatibility between the motor encoder and the Z-7010.}
	\label{fig:llc}
\end{figure}
The final signal is the clock.
When this signal is output from the FPGA, the encoder will latch the position of the motor onto the data line.
The clock signal will be $3.3V$ while the TTL $V_{ih} = 3.5V$. 
Therefore it is necessary to boost the voltage above $V_{ih}$.
This is done using a diode offset circuit as seen in figure \ref{fig:llc33to5}.
The circuit will offset the $V_{oh}$ and $V_{ol}$ by the forward voltage, $V_f$, of the diode.
Typically $V_f\approx0.7V$ and therefore $V_{oh} \approx 4.0V$ and $V_{ol} \approx 0.7V$.
These values satisfy the high/low requirements of TTL.
The pull-up resistor, $R_1$, should be sufficiently high that the output pin of the FPGA can safely sink the resulting current.
However, it should be small enough that the voltage divider created between it and the input resistance of the encoder.
Choosing $R_1=5k\Omega$ will result in a max current of $1mA$, which is well within the specifications.

\subsection{Torque Pedal Downscale}
The torque pedal is simply a variable resistor, $R_{tp}=5k\Omega$.
As the pedal is actuated the resistance changes.
In order to limit the number of power supplies needed it was decided to set the voltage across the divider to 5$V$.
The analog input of the ADC in the Z-7010 however, is limited to 0-1$V$.
Therefore it is necessary to downscale the voltage.
Intuitively this could be done using simply a voltage divider, but since the torque pedal functions as a voltage divider, the varying resistance distribution of the torque pedal would influence the voltage divider.
For this reason a voltage follower is used to isolate the two circuits.
See figure \ref{fig:torquepedaldownscale}.
The ratio of $R_1$ and $R_2$ is found similarly to that in section \ref{sec:llc}.

\begin{figure}[!h]
	\centering
	\includegraphics[width=.75\linewidth]{graphics/torque_pedal_downscale}
	\caption{Circuit used to scale the voltage from the torque pedal from $0V\rightarrow5V$ to $0V\rightarrow1V$.}
	\label{fig:torquepedaldownscale}
\end{figure}
