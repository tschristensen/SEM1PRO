\section{PMAC Motor}\label{sec:PMAC} %by Martin

%\subsection{Requirements}
%\subsection{Analysis}
%\subsection{Conclusion}

The motor used to drive the go kart is a 8 pole non-salient permanent magnet synchronous motor. Physically there is no difference between brushless DC motors, permanent magnet AC motors, and permanent magnet synchronous motors. The motor will be driven as an AC or synchronous motor (since these are the same) to produce a more consistent torque.

\subsection{Motor description}\label{sub:motor_descrpition}
The motor used is a 3 phase Wye-connected, 8 pole permanent magnet AC motor, Motenergy ME1117. This means, the rotor consists of four pairs of magnets -- one with the north side facing outward towards the stator, and one with the north side facing inward to the ferromagnetic core of the rotor. The stator consists of 24 solenoids fixed inward on a ferromagnetic outer ring. Looking at the motor from the axis side, a counter clockwise rotoration results in the go-kart going forward. The solenoids are placed so that the four phase A solenoids are placed along the horizontal and vertical axis -- this is not necessarily true, but rather a convention and a way of, and what's really important is what we measure, which will be explained further in section~\ref{sub:parking_test}. The order of the other solenoids, when going counter-clockwise is then $\bar{B}$, $C$, $\bar{A}$, $B$ and $\bar{C}$, as shown on figure~\ref{fig:motor_24p}. The solenoids $A$ correspond to terminal M1, solenoids $B$ correspond to M2, and solenoids $C$ correspond to M3. A current going into terminal M1, will then pass through all the $A$ and $\bar{A}$ solenoids in series, before reaching the star point, and continuing through $B$ and $\bar{B}$ and/or $C$ and $\bar{C}$. 


\begin{figure}[H]
	\begin{center}
		\includegraphics[width=8cm]{graphics/motor_24p}
		\label{fig:motor_24p}
		\caption{Cross section diagram of motor. Magnetic flux lines are drawn in green and cyan for a stator field being 90 electrical degrees ahead of the rotor.}
	\end{center}
\end{figure}

According to the manufacturer, the motor is rated for a 48 VDC and 300 A for a minute. This corresponds well with its 19 hp rating, which means, this DC voltage is the input voltage when driving it as a brushless DC motor, and thus a DC supply voltage of 52.8 V does not overload the motor. Through simulations, it has been noted, that the power peaks at 14 hp, at the point where controller starts saturating

For controlling the motor, it can be considered a 2 pole motor, which means there are only two magnetic fields; the rotor magnetic field generated by the permanent magnets, and the stator field generated by the solenoids. This is achieved by converting the measured mechanical angle of the rotor to electrical, and having all calculations done in this domain.

\begin{figure}[H]
	\begin{center}
		\includegraphics[width = 8cm]{graphics/motor_6p}
		\label{fig:motor_6p}
		\caption{Simplified motor model showing the stator field as green arrows}
	\end{center}
\end{figure}

The electric positions of figures~\ref{fig:motor_24p} and~\ref{fig:motor_6p} are the same, and the stator field is set 90 degrees ahead. In figure~\ref{fig:motor_6p}, the rotor magnet will always try to line up with the magnetic field lines. In the real motor, this will cause the magnetic circuits to become shorter, as the rotor turns lines up the permanent magnet poles with the opposite pole of the solenoids. By constantly placing the stator field 90 degrees ahead of the rotor, the motor will start moving. This has been compared to the classical cartoon of a carrot on a stick, tied to the back of a donkey, see figure~\ref{fig:silly_picture}

\begin{figure}[H]
	\begin{center}
		\includegraphics[width = 6cm]{graphics/carrot_stick_donkey}
		\label{fig:silly_picture}
		\caption{A childlike, but brilliant analogy of how to get the motor going}
	\end{center}
\end{figure}

Thus it is necessary to know the position of the rotor in order to determine the angle of the stator field.

\subsection{Motor parameterization}\label{sub:1117_param}
Due to the sparse information available from the manufacturer of the motor, it is necessary to measure some of the parameters of the motor. It will also be necessary to define exactly what these parameters define, since the manufacturer has not defined certain things such as what current measurement is related to torque through $K_T$.

Parameters to test are the back-EMF constant, $K_E$, armature resistance $R_a$ and armature inductance $L_a$. 

\subsubsection{Back-EMF and torque constant}\label{sub:KE-KT}
The parameters $K_E$ and $K_T$ must be the same for a PMSM motor, since they are depended on the flux produced in the permanent magnet of the rotor. Another proof, that they are the same is, that the motor converts electrical energy to mechanical energy, and electrical energy is the product of current and voltage, and (rotational) mechanical energy is the product of torque and angular velocity. A mismatch between $K_E$ and $K_T$ could imply, that the motor creates energy from nothing. 

The Back-EMF constant is defined as the amplitude of each phase-voltage divided with the mechanical angular velocity. The torque constant is defined as the peak phase-current divided by the peak-torque produced by that phase. This means that $K_E$ and $K_T$ are the same, and the torque produced is the q-current multiplied by $K_T$ and 1.5 \todo{More on this later, I have proof in simulink}. The torque constant will not be tested, because there is not a proper test bench for it.

To test the back-EMF constant, a hand-held battery-powered drill will be used to rotate the motor at a constant speed. An oscilloscope with a differential probe will be used to measure the voltage between two of the phases. Because of the high input impedance of the differential probe, there is no current and thus the armature inductance and resistance can be neglected. The peak-peak voltage and the corresponding frequency is noted in the table below:
\begin{table}[h]
\begin{center}
\begin{tabular}{r | l}
	Voltage [$V_{pp}$]  & Frequency [Hz] \\
	\hline
	28.3 & 72.13 \\
	15.3 & 38.32\\
	16.5 & 40.97\\
	17.7 & 44.9\\
	9.7 & 23.49\\
	21.8 & 54.7\\
	24.2 & 61.81
\end{tabular}
\end{center}
\caption{Measured line-line voltage at electric frequencies}
\label{tab:KE_measurements}
\end{table}

The phase amplitude of the voltage is calculated by dividing the voltages in table~\ref*{tab:KE_measurements} with $2\sqrt{3}$. The mechanical angular velocity is calculated by multiplying the frequency in table~\ref*{tab:KE_measurements} with $2\pi$, and dividing with $4$ pole pairs. The resultant voltages and velocities are divided to calculate $K_E$, and the average is calculated for these. The result is:

\begin{equation}\label{eq:K_E}
K_E = K_T = 0.0733 \frac{V s}{rad}
\end{equation}

with a spread of 1.3 percent. 

According to the datasheet, $K_T$ is 0.13. This is 1.77 times higher than the measured $K_E$, and there are two potential reasons for this. One is, that the manufacturer uses line-to-line current instead of phase current, since the difference is close to $\sqrt{3}$. Another possible reason is, that the manufacturer defines $K_T$ as the torque produced with the given q-current, in which case $K_T = 1.5K_E$, and the measured $K_T$ is about 15\% below the manufacturer's datasheet. Either way, equation~\ref{eq:K_E} will be considered true from now on. 

To confirm, that $K_E$ and $K_T$ are indeed the same, consider an ideal motor producing some torque, T, and running at a constant angular velocity, $\omega$. Losses in the Armature resistance and friction are omitted. The motor is driven by a quadrature current, I, and a quadrature voltage V. Currents and voltages along the d-axis are 0. The electrical power going into the machine must be equal to the mechanical power going out:

\begin{equation}
P_m = P_e
\label{eq:power_electrical_mechanical}
\end{equation}

Mechanical power is the product of torque and angular velocity:

\begin{equation}
P_m = \omega T
\label{eq:power_mechanical}
\end{equation}

Since there is no phase between voltages and currents, electrical power is the product of the RMS values of the current and voltage per phase, times 3:

\begin{equation}
P_e = 3\Big(\frac{1}{\sqrt{2}}I\cdot\frac{1}{\sqrt{2}}V\Big) = \frac{3}{2}IV
\label{eq:power_electrical_RMS}
\end{equation}

The angular velocity and torque can be calculated from the voltage and current, respectively, and put into equation \ref{eq:power_mechanical}:

\begin{equation}
P_m = \Big( \frac{V}{K_E} \Big) \cdot \big( 1.5 I \cdot K_T \big)
\label{eq:mechanical_power_from_electrical}
\end{equation}

Equations~\ref{eq:power_electrical_RMS}~and~\ref{eq:mechanical_power_from_electrical} must be the same, in which case, $K_E$ and $K_T$ must be the same. However it would be understandable to consider $K_T$ as 1.5 times larger than $K_E$ to incorporate the constant of 1.5

\subsubsection{Armature Resistance}\label{sub:R_a-measurements}

To test the armature resistance, a high DC current is driven through one terminal to the other. The voltage at the terminal is measured, since the large current causes a significant voltage loss in the wires leading from the power supply to the motor. The resistance can then be calculated between two terminals, the results are shown in the table below:

\begin{table}[h]
	\begin{center}
	\begin{tabular}{r | c | c | c}
		\textbf{Terminals} 	& \textbf{V}	& \textbf{I}	& \textbf{R} \\
		\hline
		\textbf{AB} 		& 0.217 V		& 17.93 A		& 12.10 $m\Omega$ \\
		\hline
		\textbf{AC}			& 0.235 V		& 17.93 A 		& 13.11 $m\Omega$ \\
		\hline
		\textbf{BC}			& 0.206 V 		& 17.96 A		& 11.47 $m\Omega$
	\end{tabular}
	\end{center}
	\label{tab:armature_resistance}
	\caption{Measured DC resistance from across terminals}
\end{table}

It is desired to know the resistance from one terminal to the common star point. That can be calculated by equation~\ref{eq:resistance_converter}

\begin{equation}
R_a = \frac{R_{AB} + R_{AC} - R_{BC}}{2}
\label{eq:resistance_converter}
\end{equation}

This gives a bit uneven armature resistances ranging from 5.2 to 6.9 $m\Omega$. This is about half the line-to-line resistance listed by the manufacturer at 13$m\Omega$. However, since the motor is a Wye connection, it makes much more sense to know the resistance between a terminal and the star-point when modeling the motor 

\subsubsection{Armature Inductance}\label{sub:L_A-measurements}
In order to measure the winding inductance of the motor, a full bridge driver is used to generate a squarewave voltage at different frequencies. The voltage is applied to the A and C terminals of the motor, so any measured voltage must be reduced by the armature resistance calculated in table~\ref{tab:armature_resistance} times the current. The motor is not moving, so there is no back-EMF. The driver can only handle 3 Amps continuously, so the test has to be done rather quickly. Current and voltage has been plotted in figure~\ref{fig:20kHz_squarewave}. Note that at 20 kHz, the current ripple is about 7 A \todo{this is peak-peak.}.

\begin{figure}[H]
	\begin{center}
		\includegraphics[width = 10cm]{graphics/inductance_measured_v_i}
		\caption{20 kHz squarewave voltage and the resulting current}
		\label{fig:20kHz_squarewave}
	\end{center}
\end{figure}

These waveforms have been measured at 5 kHz, 10 kHz, 20 kHz and 30 kHz. For this project, 20 kHz will be used, so only those plots will be shown. Due to the high sampling rate of the oscilloscope and low bit resolution, differentiating the current will yield very large spikes for short periods of time. Therefore, the current, time and voltage measurements will be periodically averaged and downsampled by a factor of 70. The derivative of the current is then calculated by:

\begin{equation}
di\_ dt[n] = \frac{i[n]-i[n-1]}{t[n]-t[n-1]}
\end{equation}

This is then the derivative of the current at the point immediately between two samples, so to synchronize the time and voltage arrays, each two samples are added and multiplied by 0.5. The resulting waveforms are plotted in figure~\ref{fig:20kHz_squarewave_di}

\begin{figure}[h]
	\begin{center}
		\includegraphics[width = 12cm]{graphics/20kHz_v_di}
		\caption{20 kHz squarewave voltage and derivative of the resulting current}
		\label{fig:20kHz_squarewave_di}
	\end{center}
\end{figure}

It appears, that there is some proportionality between the derivative of the current, and the voltage across this inductor. This has been plotted in figure~\ref{fig:20kHz_inductance}: 

\begin{figure}[H]
	\begin{center}
		\includegraphics[width = 10cm]{graphics/20kHz_inductance}
		\caption{Inductance at 20 kHz}
		\label{fig:20kHz_inductance}
	\end{center}
\end{figure}

The calculated inductance would become very large when the derivative current goes close to zero, so the plot has been limited to between 0 and 0.5 mH. The inductance inductance generally is lower than the terminal-to-terminal inductance provided by the manufacturer of 0.1 mH. The average inductance varies a bit with different frequencies, and 20 kHz is actually the lowest.

\begin{table}[H]
	\begin{center}
	\begin{tabular}{r | c | c}
		Frequency [Hz] 	& 	Average inductance [mH] & 	Current ripple [A] \\
		\hline
		5 kHz			&	0.946					&	16.83 \\
		\hline
		10 kHz			&	0.844					&	12.1 \\
		\hline
		20 kHz			&	0.632					&	7.13 \\
		\hline
		30 kHz 			&	0.879					&	5.25 
	\end{tabular}
	\label{tab:inductance_and_current_ripple}
	\caption{Averaged of measured inductance, and measured current ripple at different frequencies}
	\end{center}
\end{table}

This is the inductance from one terminal to another, so the per-phase inductance will be half, similarly to the armature resistance. The inductance determines the phase shift on the sinusoidal voltages and currents driving the motor at up to 267 Hz, and the current ripple at 20 kHz. So when modeling the motor, one must settle on a value of $L_d$ and $L_q$ that is close to the the low frequency inductance of 0.475 mH, and the higher frequency inductance of 0.315 mH. For modeling the motor, an inductance of 0.4 mH will be used for both $L_d$ and $L_q$. 

\subsection{Parking Test}\label{sub:parking_test}
The encoder is positioned on the axis of the motor in an unknown angle. Since this is not bound to change, a one time parking test will be performed, where a current will be passed into one terminal and out of the other two. This will be done with a DC power supply with current limit. The motor should position itself at one of four mechanical angles spaced 90 degrees apart. This offset will be subtracted from any angular readings used in Clark and inverse Clarke transformations. \\

The motor is parked with 30 A. Even with this current it's still possible to  The test is repeated to determine all 4 parking spots for each phase, and the angles are shown in table~\ref{tab:parking_angles}. The phases A, B and C correspond to the terminals with the names M1, M2 and M3 etched into the case of motor. 

\begin{table}[H]
	\begin{center}
		\begin{tabular}{|r | c | c | c|}
			\hline
			Quadrant & Phase A [deg] & Phase B [deg] & Phase C [deg] \\
			\hline
			First & 43.6 & 75.9 & 16.9 \\
			\hline
			Second & 132.2 & 167.3 & 101.3 \\
			\hline
			Third & 223.6 & 257.3 & 194.1 \\
			\hline
			Fourth & 312.2 & 348.8 & 279.8 \\
			\hline
		\end{tabular}
		\label{tab:parking_angles}
		\caption{Parking angles for the encoder.}
	\end{center}
\end{table}

The angles should be spaced 90 degrees apart vertically, 30 degrees apart horizontally. 
This does not hold true for all values -- both columns A and B are spaced 90 degrees apart within the precision of the encoder, but horizontally, they are spaced 34.45 degrees apart. Column C varies from 84 to 97 degrees between them, however they are on average spaced 30 degrees from Column A. This should definitely not be the case, but it's unknown why. For the controller, the Phase A readings will be considered true. The angles are then wrapped down to the first quadrant by subtracting 90, 180 and 270 degrees from second, third and fourth quadrant respectively, and the average is calculated to be $42.9 ^\circ$, corresponding to a value of 30.5 from the encoder. Whenever the rotor position is then read, 30.5 will be subtracted before anything else is done. 

It should also be noted, that the order of the phases is A, B and C, which is the opposite order of the one described in section~\ref{sub:motor_descrpition}. This means, that the encoder measures a negative speed when the go-kart is moving forward. So the electrical angle of the rotor is calculated by:

\begin{equation}
\Phi_e = -(u-30.5)\cdot \frac{2 \pi}{256} \cdot 4
\end{equation}

where u is the value sent out by the encoder. The fraction converts the value into mechanical radians, and multiplying by 4 converts mechanical to electrical.
