%!TEX root = ../main.tex
\subsection{Capacitors}\label{sub:Capacitors}
This section attempts to determine the ammount of capacitance needed in the inverter. At first the AC characteristics will be determined, and then the transient characteristics.

In order to keep the voltage of the inverter constant, a bank of capacitors will be mounted at the end of the wires going fro the battery to the inverter.
High capacity aluminum electrolytic capacitors will be used to reduce handle low frequency noise, and film capacitors will be used to handle the higher frequencies.
The shunt capacitors will act as a low pass filter together with the inductance of the wires going from the battery to the inverter. 
In order to determine the size of the capacitors, the parameters of the wire must first be determined \\

The wire acts as a resistor and an inductor in series. The resistance is determined by equation~\ref{eq:wire_resistance}. 

\begin{equation}
R_{w} = \frac{\rho l}{A}
\label{eq:wire_resistance}
\end{equation} 

where $\rho$ is electrical resistivity of the wire, l is the length, and A is the cross section area.\\
The inductance of a wire loop can be estimated by~\ref{eq:inductance_guestimate}\cite{wire_inductance}

\begin{equation}
L_w = \mu _0 l \Big(\ln \big(\tfrac{8l}{r}\big) - 2\big)
\label{eq:inductance_guestimate}
\end{equation}

where l is the length, and r is the cross section area of the wire. 
Resistance is a quite accurate, but induction is a fairly rough guesstimate.
The inductance of the wire applies to a circular wire without, whereas the wire used is strapped to the steel frame of the gokart. //

The length of the wire is approximately 4 meters m, and the cross section area is 35 \si{\milli \meter \squared}, hence the radius is 3.3 \si{\milli \meter}.
These values yield values listed in equation~\ref{eq:wire_parameters}.

\begin{equation}
\begin{split}
R_w = 1.9 \si{\milli \ohm} \\
L_w = 36 \si{\micro\henry}
\label{eq:wire_parameters}
\end{split}
\end{equation}

A capacitor size should be chosen, so that the capacitors do not resonate with the inductor.
The chosen capacitors are 8 2200~\si{\micro\farad} electrolytic capacitors, and 8 4.7~\si{\micro\farad} film capacitors.
The large capacitance of the 8 electrolytic capacitors would reduce the voltage ripple caused by the switching of the high side transistors. 
The capacitors are rated for a RMS current of 4.06 A each, which is 32.48 A in total. 
However, simulating repetitive current pulses puts the current ripple in the order of 150 A.

\subsection{Construction of the inverter}\label{sub:construction_of_inverters}