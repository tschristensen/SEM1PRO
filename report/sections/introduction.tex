%!TEX root = ../main.tex
\section{Introduction}
Power electronics is currently a thriving field in engineering.
With more and more of the energy used coming from electric sources, it becomes increasingly important to be able to convert and transmit electricity as efficiently as possible.
This process involves not only highly efficient power inverters, but also sophisticated control of these components. 
Throughout this report the process of making an inverter and control system for an electric go-kart is described.
This involves the development of an embedded system using the Zybo board as a platform as well as developing electronics for interfacing sensor equipment, manage safety procedures and supply of power to various parts of the system.
Driving a PMAC motor is a high-current task which will require the use of novel circuitry-techniques in order to minimize parasitics caused by switching throughout the inverter.
Minimizing parasitics is a crucial part of making any inverter efficient as any uncontrolled harmonics in the system will likely cause losses, thereby creating heat, increasing the size of the cooling required and, by increasing cost.
Additionally, several methods for determining the transfer function of the system in order to properly design a controller are explored.
Correctly modelling the system is an incredible advantage if high efficiency is the goal as a proper model allows the engineer to quickly, easily and perhaps most important, cheaply, test different hypothesis without relying on a physical model.
In an effort to create as accurate a model as possible, an in depth analysis of the system is done using simulations of the various parts of the system, as well as a simulation of the complete system.